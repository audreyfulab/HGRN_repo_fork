% Options for packages loaded elsewhere
\PassOptionsToPackage{unicode}{hyperref}
\PassOptionsToPackage{hyphens}{url}
%
\documentclass[
]{article}
\usepackage{amsmath,amssymb}
\usepackage{iftex}
\ifPDFTeX
  \usepackage[T1]{fontenc}
  \usepackage[utf8]{inputenc}
  \usepackage{textcomp} % provide euro and other symbols
\else % if luatex or xetex
  \usepackage{unicode-math} % this also loads fontspec
  \defaultfontfeatures{Scale=MatchLowercase}
  \defaultfontfeatures[\rmfamily]{Ligatures=TeX,Scale=1}
\fi
\usepackage{lmodern}
\ifPDFTeX\else
  % xetex/luatex font selection
\fi
% Use upquote if available, for straight quotes in verbatim environments
\IfFileExists{upquote.sty}{\usepackage{upquote}}{}
\IfFileExists{microtype.sty}{% use microtype if available
  \usepackage[]{microtype}
  \UseMicrotypeSet[protrusion]{basicmath} % disable protrusion for tt fonts
}{}
\makeatletter
\@ifundefined{KOMAClassName}{% if non-KOMA class
  \IfFileExists{parskip.sty}{%
    \usepackage{parskip}
  }{% else
    \setlength{\parindent}{0pt}
    \setlength{\parskip}{6pt plus 2pt minus 1pt}}
}{% if KOMA class
  \KOMAoptions{parskip=half}}
\makeatother
\usepackage{xcolor}
\usepackage[margin=1in]{geometry}
\usepackage{longtable,booktabs,array}
\usepackage{calc} % for calculating minipage widths
% Correct order of tables after \paragraph or \subparagraph
\usepackage{etoolbox}
\makeatletter
\patchcmd\longtable{\par}{\if@noskipsec\mbox{}\fi\par}{}{}
\makeatother
% Allow footnotes in longtable head/foot
\IfFileExists{footnotehyper.sty}{\usepackage{footnotehyper}}{\usepackage{footnote}}
\makesavenoteenv{longtable}
\usepackage{graphicx}
\makeatletter
\def\maxwidth{\ifdim\Gin@nat@width>\linewidth\linewidth\else\Gin@nat@width\fi}
\def\maxheight{\ifdim\Gin@nat@height>\textheight\textheight\else\Gin@nat@height\fi}
\makeatother
% Scale images if necessary, so that they will not overflow the page
% margins by default, and it is still possible to overwrite the defaults
% using explicit options in \includegraphics[width, height, ...]{}
\setkeys{Gin}{width=\maxwidth,height=\maxheight,keepaspectratio}
% Set default figure placement to htbp
\makeatletter
\def\fps@figure{htbp}
\makeatother
\setlength{\emergencystretch}{3em} % prevent overfull lines
\providecommand{\tightlist}{%
  \setlength{\itemsep}{0pt}\setlength{\parskip}{0pt}}
\setcounter{secnumdepth}{5}
\newlength{\cslhangindent}
\setlength{\cslhangindent}{1.5em}
\newlength{\csllabelwidth}
\setlength{\csllabelwidth}{3em}
\newlength{\cslentryspacingunit} % times entry-spacing
\setlength{\cslentryspacingunit}{\parskip}
\newenvironment{CSLReferences}[2] % #1 hanging-ident, #2 entry spacing
 {% don't indent paragraphs
  \setlength{\parindent}{0pt}
  % turn on hanging indent if param 1 is 1
  \ifodd #1
  \let\oldpar\par
  \def\par{\hangindent=\cslhangindent\oldpar}
  \fi
  % set entry spacing
  \setlength{\parskip}{#2\cslentryspacingunit}
 }%
 {}
\usepackage{calc}
\newcommand{\CSLBlock}[1]{#1\hfill\break}
\newcommand{\CSLLeftMargin}[1]{\parbox[t]{\csllabelwidth}{#1}}
\newcommand{\CSLRightInline}[1]{\parbox[t]{\linewidth - \csllabelwidth}{#1}\break}
\newcommand{\CSLIndent}[1]{\hspace{\cslhangindent}#1}
\usepackage{float}
\usepackage{multirow}
\usepackage{lastpage}
\usepackage{fancyhdr}
\pagestyle{fancy}
\usepackage{booktabs}
\usepackage{longtable}
\usepackage{array}
\usepackage{multirow}
\usepackage{wrapfig}
\usepackage{float}
\usepackage{colortbl}
\usepackage{pdflscape}
\usepackage{tabu}
\usepackage{threeparttable}
\usepackage{threeparttablex}
\usepackage[normalem]{ulem}
\usepackage{makecell}
\usepackage{xcolor}
\ifLuaTeX
  \usepackage{selnolig}  % disable illegal ligatures
\fi
\IfFileExists{bookmark.sty}{\usepackage{bookmark}}{\usepackage{hyperref}}
\IfFileExists{xurl.sty}{\usepackage{xurl}}{} % add URL line breaks if available
\urlstyle{same}
\hypersetup{
  pdftitle={HCD Simulations Write Up},
  pdfauthor={Audrey Fu Lab},
  hidelinks,
  pdfcreator={LaTeX via pandoc}}

\title{HCD Simulations Write Up}
\author{Audrey Fu Lab}
\date{2024-03-07}

\begin{document}
\maketitle

\section*{Data Simulation}

We simulate hierarchical networks in a top-down approach. We consider
several parameters of simulation such as sparsity, noise, and the
architecture of the super level graph(s), namely small-world, and
scale-free networks (Watts and Strogatz 1998; Barabási and Bonabeau
2003). We simplify our simulations by focusing on basic hierarchies with
just one or two hierarchical layers.

In each hierarchy, we start by simulating the top-level nodes in
topological order, using either a small world or scale-free network
structure (Watts and Strogatz 1998; Barabási and Bonabeau 2003). We
define origin nodes as nodes in the topological graph that have no
parental input. All origin nodes are simulated from a standard normal
distribtion. After generating an initial graph corresponding to the top
most layer of the hierarchy, we simulate the middle and bottom layers of
the hierarchy by creating groups of new nodes for each parent super-node
in the upper level(s).

Each hierarchical level contains community structure nested in the
previous layer. The number of offspring nodes generated for each parent
node in the level above is chosen from a uniform \(\text{unif}(a,b)\)
distribution. We also control the connection probabilities both within
and between the communities of each hierarchical layer. Once a
hierarchical graph is simulated we use the hierarchy to generate the
node-feature matrix which represents the expression of \(N\) genes in
\(p\) samples. The number \(N\) represents the number of nodes in the
observed (bottom) layer of the hierarchy and ranges between
\(a^{\ell+1}<N<a\times b^\ell\) where \(\ell\) represents the number of
hierarchical layers.

We consider three sets of hierarchical networks which represent varying
difficulty levels for inference:

\begin{itemize}
  \item[1.] Complex networks - - used for final simulation assessment 
  \item[2.] Intermediate networks - used for investigative model tuning and performance assessment 
  \item[3.] Simple networks - used for code implementation and debugging
\end{itemize}

\section*{Application to Intermediate Networks}

A summary of the intermediate networks can be found in \textbf{Table 1}.
These

\section*{Preliminary Findings}

\newpage
\section*{Figures}

\includegraphics{Lab_report_3_13_2024_files/figure-latex/fig1-1.pdf}

\includegraphics{Lab_report_3_13_2024_files/figure-latex/unnamed-chunk-1-1.pdf}

\begin{figure}
\centering
\includegraphics{Lab_report_3_13_2024_files/figure-latex/unnamed-chunk-2-1.pdf}
\caption{Small world graphs}
\end{figure}

\begin{figure}
\centering
\includegraphics{Lab_report_3_13_2024_files/figure-latex/unnamed-chunk-3-1.pdf}
\caption{Scale free graphs}
\end{figure}

\begin{figure}
\centering
\includegraphics{Lab_report_3_13_2024_files/figure-latex/unnamed-chunk-4-1.pdf}
\caption{random graphs}
\end{figure}

\newpage 
\section*{Tables}

\begin{table}
\centering\centering
\caption{\label{tab:unnamed-chunk-5}Summary statistics for intermediate difficulty simulated networks.}
\centering
\fontsize{10}{12}\selectfont
\fontsize{10}{12}\selectfont
\begin{tabular}[t]{>{\raggedright\arraybackslash}p{8em}llllll}
\toprule
Value & Network1 & Network2 & Network3 & Network4 & Network5 & Network6\\
\midrule
\cellcolor{gray!10}{Subgraph type} & \cellcolor{gray!10}{small world} & \cellcolor{gray!10}{small world} & \cellcolor{gray!10}{scale free} & \cellcolor{gray!10}{scale free} & \cellcolor{gray!10}{random graph} & \cellcolor{gray!10}{random graph}\\
Connection type & disc & full & disc & full & disc & full\\
\cellcolor{gray!10}{Layers} & \cellcolor{gray!10}{3} & \cellcolor{gray!10}{3} & \cellcolor{gray!10}{3} & \cellcolor{gray!10}{3} & \cellcolor{gray!10}{3} & \cellcolor{gray!10}{3}\\
Standard deviation & 0.1 & 0.1 & 0.1 & 0.1 & 0.1 & 0.1\\
\cellcolor{gray!10}{Nodes per layer} & \cellcolor{gray!10}{(5, 15, 300)} & \cellcolor{gray!10}{(5, 15, 300)} & \cellcolor{gray!10}{(5, 15, 300)} & \cellcolor{gray!10}{(5, 15, 300)} & \cellcolor{gray!10}{(5, 12, 167)} & \cellcolor{gray!10}{(5, 12, 167)}\\
\addlinespace
Edges per layer & (0, 15, 358) & (10, 25, 300) & (0, 10, 965) & (10, 20, 300) & (0, 7, 129) & (10, 17, 167)\\
\cellcolor{gray!10}{Subgraph probability} & \cellcolor{gray!10}{0.05} & \cellcolor{gray!10}{0.05} & \cellcolor{gray!10}{0.05} & \cellcolor{gray!10}{0.05} & \cellcolor{gray!10}{0.05} & \cellcolor{gray!10}{0.05}\\
Sample size & 500 & 500 & 500 & 500 & 500 & 500\\
\cellcolor{gray!10}{Modularity (top)} & \cellcolor{gray!10}{0.8} & \cellcolor{gray!10}{0.686} & \cellcolor{gray!10}{0.781} & \cellcolor{gray!10}{0.739} & \cellcolor{gray!10}{0.789} & \cellcolor{gray!10}{0.663}\\
Average node degree top & 1.193 & 1.38 & 3.217 & 3.337 & 0.772 & 0.886\\
\addlinespace
\cellcolor{gray!10}{Avg connections within top communities} & \cellcolor{gray!10}{71.6} & \cellcolor{gray!10}{73.4} & \cellcolor{gray!10}{193} & \cellcolor{gray!10}{191.6} & \cellcolor{gray!10}{25.8} & \cellcolor{gray!10}{25.8}\\
Avg. connections between top communities & 0 & 2.35 & 0 & 2.15 & 0 & 0.95\\
\cellcolor{gray!10}{Modularity (middle)} & \cellcolor{gray!10}{0.771} & \cellcolor{gray!10}{0.658} & \cellcolor{gray!10}{0.875} & \cellcolor{gray!10}{0.841} & \cellcolor{gray!10}{0.813} & \cellcolor{gray!10}{0.697}\\
Average node degree middle & 1.193 & 1.38 & 3.217 & 3.337 & 0.772 & 0.886\\
\cellcolor{gray!10}{Avg connections within middle communities} & \cellcolor{gray!10}{20} & \cellcolor{gray!10}{20} & \cellcolor{gray!10}{61.333} & \cellcolor{gray!10}{61.333} & \cellcolor{gray!10}{9.667} & \cellcolor{gray!10}{9.667}\\
\addlinespace
Avg connections between middle communities & 0.276 & 0.543 & 0.214 & 0.386 & 0.098 & 0.242\\
\bottomrule
\end{tabular}
\end{table}

\clearpage
\newpage

\begin{longtable}[]{@{}
  >{\raggedleft\arraybackslash}p{(\columnwidth - 10\tabcolsep) * \real{0.0857}}
  >{\raggedright\arraybackslash}p{(\columnwidth - 10\tabcolsep) * \real{0.1619}}
  >{\raggedright\arraybackslash}p{(\columnwidth - 10\tabcolsep) * \real{0.1714}}
  >{\raggedright\arraybackslash}p{(\columnwidth - 10\tabcolsep) * \real{0.1714}}
  >{\raggedright\arraybackslash}p{(\columnwidth - 10\tabcolsep) * \real{0.1714}}
  >{\raggedright\arraybackslash}p{(\columnwidth - 10\tabcolsep) * \real{0.2381}}@{}}
\caption{Simulation settings for intermediate difficulty networks. Each
row represents a single simulation scenario applied to all 6 simulated
networks given in Table 1}\tabularnewline
\toprule\noalign{}
\begin{minipage}[b]{\linewidth}\raggedleft
Scenario
\end{minipage} & \begin{minipage}[b]{\linewidth}\raggedright
Input Graph
\end{minipage} & \begin{minipage}[b]{\linewidth}\raggedright
Graph Recon. Loss
\end{minipage} & \begin{minipage}[b]{\linewidth}\raggedright
Attr. Recon. Loss
\end{minipage} & \begin{minipage}[b]{\linewidth}\raggedright
Modularity Weigth
\end{minipage} & \begin{minipage}[b]{\linewidth}\raggedright
Clust. Weight
\end{minipage} \\
\midrule\noalign{}
\endfirsthead
\toprule\noalign{}
\begin{minipage}[b]{\linewidth}\raggedleft
Scenario
\end{minipage} & \begin{minipage}[b]{\linewidth}\raggedright
Input Graph
\end{minipage} & \begin{minipage}[b]{\linewidth}\raggedright
Graph Recon. Loss
\end{minipage} & \begin{minipage}[b]{\linewidth}\raggedright
Attr. Recon. Loss
\end{minipage} & \begin{minipage}[b]{\linewidth}\raggedright
Modularity Weigth
\end{minipage} & \begin{minipage}[b]{\linewidth}\raggedright
Clust. Weight
\end{minipage} \\
\midrule\noalign{}
\endhead
\bottomrule\noalign{}
\endlastfoot
1 & A\_ingraph\_true & 1 = on & False (on) & 1 = on & 1 (middle), 1
(top) \\
2 & A\_corr\_no\_cutoff & 1 = on & False (on) & 1 = on & 1 (middle), 1
(top) \\
3 & A\_ingraph02 & 1 = on & False (on) & 1 = on & 1 (middle), 1 (top) \\
4 & A\_ingraph05 & 1 = on & False (on) & 1 = on & 1 (middle), 1 (top) \\
5 & A\_ingraph07 & 1 = on & False (on) & 1 = on & 1 (middle), 1 (top) \\
6 & A\_ingraph\_true & 0 = off & False (on) & 1 = on & 1 (middle), 1
(top) \\
7 & A\_corr\_no\_cutoff & 0 = off & False (on) & 1 = on & 1 (middle), 1
(top) \\
8 & A\_ingraph02 & 0 = off & False (on) & 1 = on & 1 (middle), 1
(top) \\
9 & A\_ingraph05 & 0 = off & False (on) & 1 = on & 1 (middle), 1
(top) \\
10 & A\_ingraph07 & 0 = off & False (on) & 1 = on & 1 (middle), 1
(top) \\
11 & A\_ingraph\_true & 1 = on & True (off) & 1 = on & 1 (middle), 1
(top) \\
12 & A\_corr\_no\_cutoff & 1 = on & True (off) & 1 = on & 1 (middle), 1
(top) \\
13 & A\_ingraph02 & 1 = on & True (off) & 1 = on & 1 (middle), 1
(top) \\
14 & A\_ingraph05 & 1 = on & True (off) & 1 = on & 1 (middle), 1
(top) \\
15 & A\_ingraph07 & 1 = on & True (off) & 1 = on & 1 (middle), 1
(top) \\
16 & A\_ingraph\_true & 0 = off & True (off) & 1 = on & 1 (middle), 1
(top) \\
17 & A\_corr\_no\_cutoff & 0 = off & True (off) & 1 = on & 1 (middle), 1
(top) \\
18 & A\_ingraph02 & 0 = off & True (off) & 1 = on & 1 (middle), 1
(top) \\
19 & A\_ingraph05 & 0 = off & True (off) & 1 = on & 1 (middle), 1
(top) \\
20 & A\_ingraph07 & 0 = off & True (off) & 1 = on & 1 (middle), 1
(top) \\
21 & A\_ingraph\_true & 1 = on & False (on) & 0 = off & 1 (middle), 1
(top) \\
22 & A\_corr\_no\_cutoff & 1 = on & False (on) & 0 = off & 1 (middle), 1
(top) \\
23 & A\_ingraph02 & 1 = on & False (on) & 0 = off & 1 (middle), 1
(top) \\
24 & A\_ingraph05 & 1 = on & False (on) & 0 = off & 1 (middle), 1
(top) \\
25 & A\_ingraph07 & 1 = on & False (on) & 0 = off & 1 (middle), 1
(top) \\
26 & A\_ingraph\_true & 0 = off & False (on) & 0 = off & 1 (middle), 1
(top) \\
27 & A\_corr\_no\_cutoff & 0 = off & False (on) & 0 = off & 1 (middle),
1 (top) \\
28 & A\_ingraph02 & 0 = off & False (on) & 0 = off & 1 (middle), 1
(top) \\
29 & A\_ingraph05 & 0 = off & False (on) & 0 = off & 1 (middle), 1
(top) \\
30 & A\_ingraph07 & 0 = off & False (on) & 0 = off & 1 (middle), 1
(top) \\
31 & A\_ingraph\_true & 1 = on & True (off) & 0 = off & 1 (middle), 1
(top) \\
32 & A\_corr\_no\_cutoff & 1 = on & True (off) & 0 = off & 1 (middle), 1
(top) \\
33 & A\_ingraph02 & 1 = on & True (off) & 0 = off & 1 (middle), 1
(top) \\
34 & A\_ingraph05 & 1 = on & True (off) & 0 = off & 1 (middle), 1
(top) \\
35 & A\_ingraph07 & 1 = on & True (off) & 0 = off & 1 (middle), 1
(top) \\
36 & A\_ingraph\_true & 0 = off & True (off) & 0 = off & 1 (middle), 1
(top) \\
37 & A\_corr\_no\_cutoff & 0 = off & True (off) & 0 = off & 1 (middle),
1 (top) \\
38 & A\_ingraph02 & 0 = off & True (off) & 0 = off & 1 (middle), 1
(top) \\
39 & A\_ingraph05 & 0 = off & True (off) & 0 = off & 1 (middle), 1
(top) \\
40 & A\_ingraph07 & 0 = off & True (off) & 0 = off & 1 (middle), 1
(top) \\
41 & A\_ingraph\_true & 1 = on & False (on) & 1 = on & 0.1 (middle),
1e-4 (top) \\
42 & A\_corr\_no\_cutoff & 1 = on & False (on) & 1 = on & 0.1 (middle),
1e-4 (top) \\
43 & A\_ingraph02 & 1 = on & False (on) & 1 = on & 0.1 (middle), 1e-4
(top) \\
44 & A\_ingraph05 & 1 = on & False (on) & 1 = on & 0.1 (middle), 1e-4
(top) \\
45 & A\_ingraph07 & 1 = on & False (on) & 1 = on & 0.1 (middle), 1e-4
(top) \\
46 & A\_ingraph\_true & 0 = off & False (on) & 1 = on & 0.1 (middle),
1e-4 (top) \\
47 & A\_corr\_no\_cutoff & 0 = off & False (on) & 1 = on & 0.1 (middle),
1e-4 (top) \\
48 & A\_ingraph02 & 0 = off & False (on) & 1 = on & 0.1 (middle), 1e-4
(top) \\
49 & A\_ingraph05 & 0 = off & False (on) & 1 = on & 0.1 (middle), 1e-4
(top) \\
50 & A\_ingraph07 & 0 = off & False (on) & 1 = on & 0.1 (middle), 1e-4
(top) \\
51 & A\_ingraph\_true & 1 = on & True (off) & 1 = on & 0.1 (middle),
1e-4 (top) \\
52 & A\_corr\_no\_cutoff & 1 = on & True (off) & 1 = on & 0.1 (middle),
1e-4 (top) \\
53 & A\_ingraph02 & 1 = on & True (off) & 1 = on & 0.1 (middle), 1e-4
(top) \\
54 & A\_ingraph05 & 1 = on & True (off) & 1 = on & 0.1 (middle), 1e-4
(top) \\
55 & A\_ingraph07 & 1 = on & True (off) & 1 = on & 0.1 (middle), 1e-4
(top) \\
56 & A\_ingraph\_true & 0 = off & True (off) & 1 = on & 0.1 (middle),
1e-4 (top) \\
57 & A\_corr\_no\_cutoff & 0 = off & True (off) & 1 = on & 0.1 (middle),
1e-4 (top) \\
58 & A\_ingraph02 & 0 = off & True (off) & 1 = on & 0.1 (middle), 1e-4
(top) \\
59 & A\_ingraph05 & 0 = off & True (off) & 1 = on & 0.1 (middle), 1e-4
(top) \\
60 & A\_ingraph07 & 0 = off & True (off) & 1 = on & 0.1 (middle), 1e-4
(top) \\
61 & A\_ingraph\_true & 1 = on & False (on) & 0 = off & 0.1 (middle),
1e-4 (top) \\
62 & A\_corr\_no\_cutoff & 1 = on & False (on) & 0 = off & 0.1 (middle),
1e-4 (top) \\
63 & A\_ingraph02 & 1 = on & False (on) & 0 = off & 0.1 (middle), 1e-4
(top) \\
64 & A\_ingraph05 & 1 = on & False (on) & 0 = off & 0.1 (middle), 1e-4
(top) \\
65 & A\_ingraph07 & 1 = on & False (on) & 0 = off & 0.1 (middle), 1e-4
(top) \\
66 & A\_ingraph\_true & 0 = off & False (on) & 0 = off & 0.1 (middle),
1e-4 (top) \\
67 & A\_corr\_no\_cutoff & 0 = off & False (on) & 0 = off & 0.1
(middle), 1e-4 (top) \\
68 & A\_ingraph02 & 0 = off & False (on) & 0 = off & 0.1 (middle), 1e-4
(top) \\
69 & A\_ingraph05 & 0 = off & False (on) & 0 = off & 0.1 (middle), 1e-4
(top) \\
70 & A\_ingraph07 & 0 = off & False (on) & 0 = off & 0.1 (middle), 1e-4
(top) \\
71 & A\_ingraph\_true & 1 = on & True (off) & 0 = off & 0.1 (middle),
1e-4 (top) \\
72 & A\_corr\_no\_cutoff & 1 = on & True (off) & 0 = off & 0.1 (middle),
1e-4 (top) \\
73 & A\_ingraph02 & 1 = on & True (off) & 0 = off & 0.1 (middle), 1e-4
(top) \\
74 & A\_ingraph05 & 1 = on & True (off) & 0 = off & 0.1 (middle), 1e-4
(top) \\
75 & A\_ingraph07 & 1 = on & True (off) & 0 = off & 0.1 (middle), 1e-4
(top) \\
76 & A\_ingraph\_true & 0 = off & True (off) & 0 = off & 0.1 (middle),
1e-4 (top) \\
77 & A\_corr\_no\_cutoff & 0 = off & True (off) & 0 = off & 0.1
(middle), 1e-4 (top) \\
78 & A\_ingraph02 & 0 = off & True (off) & 0 = off & 0.1 (middle), 1e-4
(top) \\
79 & A\_ingraph05 & 0 = off & True (off) & 0 = off & 0.1 (middle), 1e-4
(top) \\
80 & A\_ingraph07 & 0 = off & True (off) & 0 = off & 0.1 (middle), 1e-4
(top) \\
\end{longtable}

\bibliographystyle{unsrt}
    \bibliography{C:/Users/Bruin/Desktop/Research Assistantship/Thesis Proposal Defense/proposal_references.bib}

\newpage
\section*{References}

\hypertarget{refs}{}
\begin{CSLReferences}{1}{0}
\leavevmode\vadjust pre{\hypertarget{ref-barabasi2003scale}{}}%
Barabási, Albert-László, and Eric Bonabeau. 2003. {``Scale-Free
Networks.''} \emph{Scientific American} 288 (5): 60--69.

\leavevmode\vadjust pre{\hypertarget{ref-watts1998collective}{}}%
Watts, Duncan J, and Steven H Strogatz. 1998. {``Collective Dynamics of
`Small-World'networks.''} \emph{Nature} 393 (6684): 440--42.

\end{CSLReferences}

\end{document}
