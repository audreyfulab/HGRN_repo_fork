\documentclass[a4paper,12pt]{article}
\usepackage [left=25.4mm,top=25.4mm]{geometry}
\usepackage{amsmath}
\usepackage{amssymb}
\usepackage{graphicx}
%\usepackage{apacite}
\usepackage{url}
\usepackage{subfig}
\usepackage{csvsimple}
\usepackage{float}
\usepackage{lineno}
\usepackage[affil-it]{authblk}
\usepackage{setspace}
\usepackage{makecell} 
\usepackage{tikz}
\usepackage{csvsimple}
\usepackage{newfloat}
\usepackage{xcolor}
\usepackage{tabularx,booktabs}
\usepackage{multirow}
\usepackage{multicol}
\usepackage{array}
\usepackage{tocbasic}
\usepackage{sectsty}

\sectionfont{\fontsize{12}{15}\selectfont}
\chapterfont{\fontsize{14}{15}\selectfont}
\subsectionfont{\fontsize{10}{15}\selectfont}

\newcommand\hcancel[2][black]{\setbox0=\hbox{$#2$}%
	\rlap{\raisebox{.45\ht0}{\textcolor{#1}{\rule{\wd0}{1pt}}}}#2} 
\newcommand{\forceindent}{\leavevmode{\parindent=2em\indent}}

\DeclareFloatingEnvironment[name={Supplementary Figure},fileext=lsf,listname={List of Supplementary Figures}]{suppfigure}
\DeclareFloatingEnvironment[name={Supplementary Table},fileext=lsf,listname={List of Supplementary Tables}]{supptable}
\DeclareFloatingEnvironment[name={Supplementary Material},fileext=lsf,listname={List of Supplementary Material}]{suppmat}

\newcolumntype{P}[1]{>{\centering\arraybackslash}p{#1}}
\newcolumntype{M}[1]{>{\centering\arraybackslash}m{#1}}

\DeclareMathOperator*{\argmax}{arg\,max}
\DeclareMathOperator*{\argmin}{arg\,min}

\newcommand{\indep}{\perp \!\!\! \perp}
\renewcommand*\contentsname{TABLE OF CONTENTS}
\renewcommand{\listfigurename}{LIST OF FIGURES}
\renewcommand{\listtablename}{LIST OF TABLES}

\renewcommand{\arraystretch}{1.5}
\begin{document}
	\begin{titlepage}
		\title{Modularity Examples}
		\author[1]{Jarred M. Kvamme}
		\affil[1]{Department of Bioinformatics and Computational Biology - University of Idaho}
		\maketitle
	\end{titlepage}
	
	
	\newpage
	\tableofcontents{}
	\addcontentsline{toc}{section}{LIST OF TABLES}
	\addcontentsline{toc}{section}{LIST OF FIGURES}
	\listoftables
	\listoffigures
	\newpage
	
	\section{Case 1: Two completely disconnected communities}
	In this first case, we explore how modularity is influenced by community structure exploring an elementary example consisting of two completely disconnected sub-networks of $5$ nodes each. These two sub-networks correspond to two communities. We focus on two examples: \textbf{Example 1} consists of calculating the modularity when the nodes in sub-networks assigned to their correct communities and \textbf{Example 2} consists of calculating the modularity when all the nodes are assigned to a single community \textbf{Figure \ref{fig:case1}}. For all examples, we use the following formulation of the modularity 
	\[Q = \sum_{k=1}^K\frac{\Sigma_{in,k}}{2m} - \left(\frac{\Sigma_{tot,k}}{2m}\right)^2 \]
	where $\Sigma_{in,k}$ is the total number of edges within the $k^{th}$ community (multiplied by 2), $\sigma_{tot,k}$ is the total number of edges incident to each node in community $k$, and $m$ is the total number of edges in the network
	\begin{figure}[H]
		\centering
		\caption{Case 1 graph}
		\includegraphics[scale=0.65]{C:/Users/Bruin/Documents/GitHub/HGRN_repo/Modularity/case1_2clusts.pdf}
		\label{fig:case1}
	\end{figure}
	\subsection*{Example 1: Modularity for two communities of 5 nodes}
	We start by computing the modularity for community one (teal). The structure of each sub-network is identical which greatly simplified the calculations. For community 1, the number of edges within is $\Sigma_{in,1} = 2\times5$, the number of incident edges is $\Sigma_{tot,1} = 10$ and $m = 10$. Therefore the modularity of community 1 is 
	\[Q_1 = \frac{10}{2(10)} - \frac{10^2}{4(100)} = 0.25\]
	similarly for community 2 we have 
	\[Q_2 = \frac{10}{2(10)} - \frac{10^2}{4(100)} = 0.25\]
	Therefore the total modularity is the sum of the modularity of each community:
	
	\[ Q = \sum_{k=1}^2 Q_k = 0.5\]
	\subsection*{Example 2: Modularity for a single community of 10 nodes}
	In our second example we compute the modularity when all of the nodes are grouped into a single community. The is gives $\Sigma_{in,1} = 2(10) = 20$, $\Sigma_{tot,1} = 2(10) = 20$,  $m = 10$ which gives:
	
	\[ Q = \frac{20}{2(10)}-\frac{20^2}{4(100)} = 1-1 = 0 \]
	
	

	\section{Case 2: A single network divided into 2 communities}
	In this second case, we explore how modularity is influenced by community structure through another elementary example in which we have a single ring-structured community which is divided evenly into two sub-networks of $5$ nodes each \textbf{\textbf{Figure \ref{fig:case2}}}. We again focus on two examples: \textbf{Example 1} calculating the modularity when the nodes are divided into two communities and \textbf{Example 2} calculating the modularity when all the nodes in the ring are assigned to a single community. 
	\begin{figure}[H]
		\centering
		\caption{Case 2 graph}
		\includegraphics[scale=0.65]{C:/Users/Bruin/Documents/GitHub/HGRN_repo/Modularity/on_net_multi_comms.pdf}
		\label{fig:case2}
	\end{figure}
	
	
	\subsection*{Example 1: Modularity for a ring divided into two communities of 5 nodes}
		We start by computing the individual modularities for each community. Again structure of each sub-network is identical making the calculations simple. For community 1, the number of edges within is $\Sigma_{in,1} = 2\times4$, the number of incident edges is $\Sigma_{tot,1} = 10$ and $m = 10$. Therefore the modularity of community 1 is 
		\[Q_1 = \frac{8}{2(10)} - \frac{10^2}{4(20)} = 0.15\]
		similarly for community 2 we have 
		\[Q_2 = \frac{10}{2(20)} - \frac{10^2}{4(20)} = 0.15\]
		Therefore the total modularity is the sum of the modularity of each community:
	
	\[ Q = \sum_{k=1}^2 Q_k = 0.3\]
	\newpage
	\subsection*{Example 2: Modularity for all nodes in a single community}
		In our second example we compute the modularity when all of the nodes are grouped into a single community. The is gives $\Sigma_{in,1} = 2(10) = 20$, $\Sigma_{tot,1} = 2(10) = 20$,  $m = 10$ which gives:
	
		\[ Q = \frac{20}{2(10)}-\frac{20^2}{4(100)} = 1-1 = 0 \]
	
	
	
	
	
	
	
	
	
	
	
	
	
	
	
	
	
	
	
	
	
	

	\section{Case 3: The case of a singleton community}
	In this second example, we explore how modularity is affected by the presence of a singleton community \textbf{Figure \ref{fig:case3}}. In \textbf{Example 1} we calculate the modularity when there are two communities where one community consists of a single isolated node (singleton) and the other community consists of a community of 5 nodes. In \textbf{Example 2} we expand on \textbf{Example 1} by allowing the singleton community to have a single edge connecting it the primary network of 5 nodes \textbf{Figure \ref{fig:case3b}}. Lastly, in \textbf{Examples 3-4} we calculate the modularity for \textbf{Examples 1-2} when grouping all nodes into a single community. 
	
	\begin{figure}[H]
		\centering
		\caption{Case 3 graph}
		\includegraphics[scale=0.65]{C:/Users/Bruin/Documents/GitHub/HGRN_repo/Modularity/case2_singleton.pdf}
		\label{fig:case3}
	\end{figure}
	
	\subsection*{Example 1: Modularity for a ring divided into two communities of 5 nodes}
	We start by computing the individual modularities for each community. For the singleton community, the number of edges within is $\Sigma_{in,1} = 0$, the number of incident edges is $\Sigma_{tot,1} = 5$ and $m = 5$. Therefore the modularity of the singleton community is 
	\[Q_1 = \frac{0}{2(5)} - \frac{0^2}{4(25)} = 0\]
	For the second community of five nodes in a ring structure, we have 
	\[Q_2 = \frac{10}{2(5)} - \frac{10^2}{4(25)} = 0\]
	Therefore the total modularity is the sum of the modularity of each community:
	
	\[ Q = \sum_{k=1}^K Q_k = 0\]
	\subsection*{Example 2: Modularity for all nodes in a single community}
	In our second example we compute the modularity similar to \textbf{Example 1} but now the singleton community share a single edge to the other community of give nodes. The choice of which node the edge connects to is arbitrary and does not effect the modularity.  
	
	\begin{figure}[H]
		\centering
		\caption{Case 3(b) graph}
		\includegraphics[scale=0.65]{C:/Users/Bruin/Documents/GitHub/HGRN_repo/Modularity/singleton_with_edge.pdf}
		\label{fig:case3b}
	\end{figure}
	
	For the singleton community, the number of edges within is still $\Sigma_{in,1} = 0$, the number of incident edges is now $\Sigma_{tot,1} = 6$ and $m = 6$. Therefore the modularity of the singleton community is 
	\[Q_1 = \frac{0}{2(6)} - \frac{1^2}{4(36)} = -0.00694\]
	The same value as before. For the second community of five nodes in a ring structure, we have 
	\[Q_2 = \frac{10}{2(6)} - \frac{11^2}{4(100)} = -0.00694\]
	Therefore the total modularity is the sum of the modularity of each community:
	
	\[ Q = \sum_{k=1}^K Q_k = -0.01389\]
	
	\subsection*{Example 3: Example 1 with a single community}
	In this third example we compute the modularity when all of the nodes from the graph in \textbf{Example 1} are grouped into a single community. The is gives $\Sigma_{in} = 2(5) = 10$, $\Sigma_{tot} = 2(5) = 10$,  $m = 5$ which gives:
	
	\[ Q = \frac{10}{2(5)}-\frac{10^2}{4(25)} = 1-1 = 0 \]
	\subsection*{Example 4: Example 2 with a single community}
	Lastly, for the fourth example we compute the modularity when all of the nodes from the graph in \textbf{Example 2} are grouped into a single community. The is gives $\Sigma_{in} = 2(6) = 12$, $\Sigma_{tot} = 2(6) = 12$,  $m = 6$ which gives:
	
	\[ Q = \frac{12}{2(6)}-\frac{12^2}{4(36)} = 1-1 = 0 \]
	
	
	
	\newpage
	\appendix
	
	\section*{Case 1 In Linear Form}
	Please refer to the network in \textbf{Figure \ref{fig:case1}}:
	
	\subsection*{Example 1: Modularity for two communities of 5 nodes}
	Let $\bf{C}$ be the the community assignment matrix 
	\[ {\bf C} = \begin{bmatrix}1 & 0 \\ 1 & 0 \\ 1 & 0 \\ 1 & 0 \\ 1 & 0 \\ 1 & 0 \\ 0 & 1 \\ 0 & 1 \\ 0 & 1 \\ 0 & 1 \\ 0 & 1 \\
	\end{bmatrix}\] 
	
	Let $\bf A$ be the graph adjacency matrix
	\[ {\bf A} = \begin{bmatrix}
		0& 1& 0& 0& 1& 0& 0& 0& 0& 0\\
		1& 0& 1& 0& 0& 0& 0& 0& 0& 0\\
		0& 1& 0& 1& 0& 0& 0& 0& 0& 0\\
		0& 0& 1& 0& 1& 0& 0& 0& 0& 0\\
		1& 0& 0& 1& 0& 0& 0& 0& 0& 0\\
		0& 0& 0& 0& 0& 0& 1& 0& 0& 1\\
		0& 0& 0& 0& 0& 1& 0& 1& 0& 0\\
		0& 0& 0& 0& 0& 0& 1& 0& 1& 0\\
		0& 0& 0& 0& 0& 0& 0& 1& 0& 1\\
		0& 0& 0& 0& 0& 1& 0& 0& 1& 0
	\end{bmatrix}\] 
	
	Further let $\bf B$ be the modularity matrix defined as 
	
	\[ {\bf B} = {\bf A} - \frac{1}{2m}{\bf r}\otimes {\bf r} \]
	
	where 
	
	\[ {\bf r} = \begin{bmatrix}
		0& 1& 0& 0& 1& 0& 0& 0& 0& 0\\
		1& 0& 1& 0& 0& 0& 0& 0& 0& 0\\
		0& 1& 0& 1& 0& 0& 0& 0& 0& 0\\
		0& 0& 1& 0& 1& 0& 0& 0& 0& 0\\
		1& 0& 0& 1& 0& 0& 0& 0& 0& 0\\
		0& 0& 0& 0& 0& 0& 1& 0& 0& 1\\
		0& 0& 0& 0& 0& 1& 0& 1& 0& 0\\
		0& 0& 0& 0& 0& 0& 1& 0& 1& 0\\
		0& 0& 0& 0& 0& 0& 0& 1& 0& 1\\
		0& 0& 0& 0& 0& 1& 0& 0& 1& 0
	\end{bmatrix} \cdot \begin{bmatrix}
		1 & \\ 1 & \\ 1 & \\ 1 & \\ 1 & \\ 1 & \\ 1 & \\ 1 & \\ 1 & \\ 1 & \\
	\end{bmatrix} = \begin{bmatrix}
		2 & \\ 2 & \\ 2 & \\ 2 & \\ 2 & \\ 2 & \\ 2 & \\ 2 & \\ 2 & \\ 2 & \\
	\end{bmatrix} \]
	
	therefore
	
	\[ {\bf r}\otimes {\bf r} = \begin{bmatrix}
		4& 4& 4& 4& 4& 4& 4& 4& 4& 4\\
		4& 4& 4& 4& 4& 4& 4& 4& 4& 4\\
		4& 4& 4& 4& 4& 4& 4& 4& 4& 4\\
		4& 4& 4& 4& 4& 4& 4& 4& 4& 4\\
		4& 4& 4& 4& 4& 4& 4& 4& 4& 4\\
		4& 4& 4& 4& 4& 4& 4& 4& 4& 4\\
		4& 4& 4& 4& 4& 4& 4& 4& 4& 4\\
		4& 4& 4& 4& 4& 4& 4& 4& 4& 4\\
		4& 4& 4& 4& 4& 4& 4& 4& 4& 4\\
		4& 4& 4& 4& 4& 4& 4& 4& 4& 4
	\end{bmatrix}\]
	
	we also have that 
	
	\[ m = \frac{1}{2} \sum_i \sum_j {\bf A}_{i,j} = 10\]
	
	therefore the modularity matrix can be expressed as 
	\[ {\bf B} = \begin{bmatrix}
		-\frac{4}{20}& \frac{16}{20}& -\frac{4}{20}& -\frac{4}{20}& \frac{16}{20}& -\frac{4}{20}& -\frac{4}{20}& -\frac{4}{20}& -\frac{4}{20}& -\frac{4}{20}\\
		\frac{16}{20}& -\frac{4}{20}& \frac{16}{20}& -\frac{4}{20}& -\frac{4}{20}& -\frac{4}{20}& -\frac{4}{20}& -\frac{4}{20}& -\frac{4}{20}& -\frac{4}{20}\\
		-\frac{4}{20}& \frac{16}{20}& -\frac{4}{20}& \frac{16}{20}& -\frac{4}{20}& -\frac{4}{20}& -\frac{4}{20}& -\frac{4}{20}& -\frac{4}{20}& -\frac{4}{20}\\
		-\frac{4}{20}& -\frac{4}{20}& \frac{16}{20}& -\frac{4}{20}& \frac{16}{20}& -\frac{4}{20}& -\frac{4}{20}& -\frac{4}{20}& -\frac{4}{20}& -\frac{4}{20}\\
		\frac{16}{20}& -\frac{4}{20}& -\frac{4}{20}& \frac{16}{20}& -\frac{4}{20}& -\frac{4}{20}& -\frac{4}{20}& -\frac{4}{20}& -\frac{4}{20}& -\frac{4}{20}\\
		-\frac{4}{20}& -\frac{4}{20}& -\frac{4}{20}& -\frac{4}{20}& -\frac{4}{20}& -\frac{4}{20}& \frac{16}{20}& -\frac{4}{20}& -\frac{4}{20}& \frac{16}{20}\\
		-\frac{4}{20}& -\frac{4}{20}& -\frac{4}{20}& -\frac{4}{20}& -\frac{4}{20}& \frac{16}{20}& -\frac{4}{20}& \frac{16}{20}& -\frac{4}{20}& -\frac{4}{20}\\
		-\frac{4}{20}& -\frac{4}{20}& -\frac{4}{20}& -\frac{4}{20}& -\frac{4}{20}& -\frac{4}{20}& \frac{16}{20}& -\frac{4}{20}& \frac{16}{20}& -\frac{4}{20}\\
		-\frac{4}{20}& -\frac{4}{20}& -\frac{4}{20}& -\frac{4}{20}& -\frac{4}{20}& -\frac{4}{20}& -\frac{4}{20}& \frac{16}{20}& -\frac{4}{20}& \frac{16}{20}\\
		-\frac{4}{20}& -\frac{4}{20}& -\frac{4}{20}& -\frac{4}{20}& -\frac{4}{20}& \frac{16}{20}& -\frac{4}{20}& -\frac{4}{20}& \frac{16}{20}& -\frac{4}{20}
	\end{bmatrix} \]
	
	We compute the modularity using the following equation
	
	\[M = \frac{1}{4m}Tr({\bf C^T B C}) \]
	
	where 
	\[ {\bf C^T B C} = \] 
	\[\small \begin{bmatrix}
		1 & 1 & 1 & 1 & 1 & 0 & 0 & 0 & 0 & 0 \\
		0 & 0 & 0 & 0 & 0 & 1 & 1 & 1 & 1 & 1 \\
	\end{bmatrix}\cdot  \begin{bmatrix}
		-\frac{4}{20}& \frac{16}{20}& -\frac{4}{20}& -\frac{4}{20}& \frac{16}{20}& -\frac{4}{20}& -\frac{4}{20}& -\frac{4}{20}& -\frac{4}{20}& -\frac{4}{20}\\
		\frac{16}{20}& -\frac{4}{20}& \frac{16}{20}& -\frac{4}{20}& -\frac{4}{20}& -\frac{4}{20}& -\frac{4}{20}& -\frac{4}{20}& -\frac{4}{20}& -\frac{4}{20}\\
		-\frac{4}{20}& \frac{16}{20}& -\frac{4}{20}& \frac{16}{20}& -\frac{4}{20}& -\frac{4}{20}& -\frac{4}{20}& -\frac{4}{20}& -\frac{4}{20}& -\frac{4}{20}\\
		-\frac{4}{20}& -\frac{4}{20}& \frac{16}{20}& -\frac{4}{20}& \frac{16}{20}& -\frac{4}{20}& -\frac{4}{20}& -\frac{4}{20}& -\frac{4}{20}& -\frac{4}{20}\\
		\frac{16}{20}& -\frac{4}{20}& -\frac{4}{20}& \frac{16}{20}& -\frac{4}{20}& -\frac{4}{20}& -\frac{4}{20}& -\frac{4}{20}& -\frac{4}{20}& -\frac{4}{20}\\
		-\frac{4}{20}& -\frac{4}{20}& -\frac{4}{20}& -\frac{4}{20}& -\frac{4}{20}& -\frac{4}{20}& \frac{16}{20}& -\frac{4}{20}& -\frac{4}{20}& \frac{16}{20}\\
		-\frac{4}{20}& -\frac{4}{20}& -\frac{4}{20}& -\frac{4}{20}& -\frac{4}{20}& \frac{16}{20}& -\frac{4}{20}& \frac{16}{20}& -\frac{4}{20}& -\frac{4}{20}\\
		-\frac{4}{20}& -\frac{4}{20}& -\frac{4}{20}& -\frac{4}{20}& -\frac{4}{20}& -\frac{4}{20}& \frac{16}{20}& -\frac{4}{20}& \frac{16}{20}& -\frac{4}{20}\\
		-\frac{4}{20}& -\frac{4}{20}& -\frac{4}{20}& -\frac{4}{20}& -\frac{4}{20}& -\frac{4}{20}& -\frac{4}{20}& \frac{16}{20}& -\frac{4}{20}& \frac{16}{20}\\
		-\frac{4}{20}& -\frac{4}{20}& -\frac{4}{20}& -\frac{4}{20}& -\frac{4}{20}& \frac{16}{20}& -\frac{4}{20}& -\frac{4}{20}& \frac{16}{20}& -\frac{4}{20}
	\end{bmatrix} \cdot 
	\begin{bmatrix}
		1 & 0 \\ 
		1 & 0 \\
		1 & 0 \\
		1 & 0 \\ 
		1 & 0 \\ 
		1 & 0 \\ 
		0 & 1 \\ 
		0 & 1 \\ 
		0 & 1 \\ 
		0 & 1 \\ 
		0 & 1 
	\end{bmatrix}\]
	
	\[\small =\begin{bmatrix}
		1&  1&  1&  1&  1& -1& -1& -1& -1& -1 \\
		-1& -1& -1& -1& -1&  1&  1&  1&  1&  1 \\
	\end{bmatrix} \cdot 
	\begin{bmatrix}
		1 & 0 \\ 
		1 & 0 \\
		1 & 0 \\
		1 & 0 \\ 
		1 & 0 \\ 
		1 & 0 \\ 
		0 & 1 \\ 
		0 & 1 \\ 
		0 & 1 \\ 
		0 & 1 \\ 
		0 & 1 
	\end{bmatrix}\]
	
	
	\[ =\begin{bmatrix}
		5 & -5 \\
		-5 & 5 \\
	\end{bmatrix} \]
	
	Therefore the modularity equals 
	
	\[ M = \frac{1}{40} \times Tr\left( \begin{bmatrix}
		5 & -5 \\
		-5 & 5 \\
	\end{bmatrix} \right) \]
	
	\[ = \frac{1}{40} \times 10 = 0.25\]
	
	
	\subsection*{Example 2: Modularity for a single community of 10 nodes}
	In our second example we compute the modularity when all of the nodes are grouped into a single cluster. The only computation which we much change is ${\bf C^T B C}$. In this scenario, the community assignment matrix is as follows
	
	\[ {\bf C} = \begin{bmatrix}1 & 0 \\ 1 & 0 \\ 1 & 0 \\ 1 & 0 \\ 1 & 0 \\ 1 & 0 \\ 1 & 0 \\ 1 & 0 \\ 1 & 0 \\ 1 & 0 \\ 1 & 0 \\
	\end{bmatrix}\] 
	
	
	\[ {\bf C^T B C} = \] 
	\[\small \begin{bmatrix}
		1 & 1 & 1 & 1 & 1 & 1 & 1 & 1 & 1 & 1 \\
		0 & 0 & 0 & 0 & 0 & 0 & 0 & 0 & 0 & 0 \\
	\end{bmatrix}\cdot  \begin{bmatrix}
		-\frac{4}{20}& \frac{16}{20}& -\frac{4}{20}& -\frac{4}{20}& \frac{16}{20}& -\frac{4}{20}& -\frac{4}{20}& -\frac{4}{20}& -\frac{4}{20}& -\frac{4}{20}\\
		\frac{16}{20}& -\frac{4}{20}& \frac{16}{20}& -\frac{4}{20}& -\frac{4}{20}& -\frac{4}{20}& -\frac{4}{20}& -\frac{4}{20}& -\frac{4}{20}& -\frac{4}{20}\\
		-\frac{4}{20}& \frac{16}{20}& -\frac{4}{20}& \frac{16}{20}& -\frac{4}{20}& -\frac{4}{20}& -\frac{4}{20}& -\frac{4}{20}& -\frac{4}{20}& -\frac{4}{20}\\
		-\frac{4}{20}& -\frac{4}{20}& \frac{16}{20}& -\frac{4}{20}& \frac{16}{20}& -\frac{4}{20}& -\frac{4}{20}& -\frac{4}{20}& -\frac{4}{20}& -\frac{4}{20}\\
		\frac{16}{20}& -\frac{4}{20}& -\frac{4}{20}& \frac{16}{20}& -\frac{4}{20}& -\frac{4}{20}& -\frac{4}{20}& -\frac{4}{20}& -\frac{4}{20}& -\frac{4}{20}\\
		-\frac{4}{20}& -\frac{4}{20}& -\frac{4}{20}& -\frac{4}{20}& -\frac{4}{20}& -\frac{4}{20}& \frac{16}{20}& -\frac{4}{20}& -\frac{4}{20}& \frac{16}{20}\\
		-\frac{4}{20}& -\frac{4}{20}& -\frac{4}{20}& -\frac{4}{20}& -\frac{4}{20}& \frac{16}{20}& -\frac{4}{20}& \frac{16}{20}& -\frac{4}{20}& -\frac{4}{20}\\
		-\frac{4}{20}& -\frac{4}{20}& -\frac{4}{20}& -\frac{4}{20}& -\frac{4}{20}& -\frac{4}{20}& \frac{16}{20}& -\frac{4}{20}& \frac{16}{20}& -\frac{4}{20}\\
		-\frac{4}{20}& -\frac{4}{20}& -\frac{4}{20}& -\frac{4}{20}& -\frac{4}{20}& -\frac{4}{20}& -\frac{4}{20}& \frac{16}{20}& -\frac{4}{20}& \frac{16}{20}\\
		-\frac{4}{20}& -\frac{4}{20}& -\frac{4}{20}& -\frac{4}{20}& -\frac{4}{20}& \frac{16}{20}& -\frac{4}{20}& -\frac{4}{20}& \frac{16}{20}& -\frac{4}{20}
	\end{bmatrix} \cdot 
	\begin{bmatrix}
		1 & 0 \\ 
		1 & 0 \\
		1 & 0 \\
		1 & 0 \\ 
		1 & 0 \\ 
		1 & 0 \\ 
		1 & 0 \\ 
		1 & 0 \\ 
		1 & 0 \\ 
		1 & 0 \\ 
		1 & 0 
	\end{bmatrix}\]
	
	\[\small =\begin{bmatrix}
		0&  0&  0&  0&  0& 0& 0& 0& 0& 0 \\
		0& 0& 0& 0& 0& 0& 0& 0& 0&  0 \\
	\end{bmatrix} \cdot 
	\begin{bmatrix}
		1 & 0 \\ 
		1 & 0 \\
		1 & 0 \\
		1 & 0 \\ 
		1 & 0 \\ 
		1 & 0 \\ 
		1 & 0 \\ 
		1 & 0 \\ 
		1 & 0 \\ 
		1 & 0 \\ 
		1 & 0 
	\end{bmatrix}\]
	
	
	\[ =\begin{bmatrix}
		0 & 0 \\
		0 & 0 \\
	\end{bmatrix} \]
	
	\[M = \frac{1}{40} \times 0 = 0\]
	
	
	\section{Case 3}
	\[ {\bf C} = \begin{bmatrix}1 & 0 \\ 1 & 0 \\ 1 & 0 \\ 1 & 0 \\ 1 & 0 \\ 0 & 1 \\	\end{bmatrix}\] 
	
	Let $\bf A$ be the graph adjacency matrix
	\[ {\bf A} = \begin{bmatrix}
		0& 1& 0& 0& 1& 0 \\
		1& 0& 1& 0& 0& 0 \\
		0& 1& 0& 1& 0& 0 \\
		0& 0& 1& 0& 1& 0 \\
		1& 0& 0& 1& 0& 0 \\
		0& 0& 0& 0& 0& 0 \\
	\end{bmatrix}\] 
	
	
	
	
	
	
	
	
	
	
	
	
	
	
	
	
	
	
	\clearpage
	\newpage
	
	\bibliographystyle{unsrt}
	\bibliography{pseudocode_bibs}
	
	
	
	
	
	
\end{document}